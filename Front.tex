%\copyrightpage[...]{...}		% Optional, comment out or delete if undesired

\begin{abstract}

This is the abstract. Sentence two. Formatting is easy when you use \LaTeX and it's easy to control. It excels in the math environement but tables can sometimes require more effort. Fortunately, it's open source (i.e. free), platform independent, and there's a big user community. There's a list of resources at the end.


\end{abstract}

%\begin{layabstract}{...}	% Replace the ... with the list of keywords
% Lay abstract text here, either typed in directly or included using an `\input{}' command
%\end{layabstract}

% The optional preface, dedication, and acknowledgements environments are included similar to the abstract environment

%%%%%%%%%%%%%%%%%%%%%%%%%%%% Preface %%%%%%%%%%%%%%%%%%%%%%%%%%%%%%%%%
%\begin{preface}
% Preface text here
%\end{preface} 

%%%%%%%%%%%%%%%%%%%%%%%%%%% Dedication %%%%%%%%%%%%%%%%%%%%%%%%%%%%%%%
%\begin{dedication}
% Dedication text here
%\end{dedication}

%%%%%%%%%%%%%%%%%%%%%%%% Acknowledgements %%%%%%%%%%%%%%%%%%%%%%%%%%%%
%\begin{acknowledgements}
% Acknowledgments text here
%\end{acknowledgements}


%%%%%%%%%%%%%%%%%%%%%%%%%%%%%%%%%%%%%%%%%%%%%%%%%%%%%%%%%%%%%%%%%%%%%%
% Commands for the required lists
\setcounter{page}{2}
\tableofcontents
\listoftables				% Include only if there are tables in the thesis
\listoffigures				% Include only if there are figures in the thesis


%%%%%%%%%%%%%%%%%%%%%%%%%%%%%%%%%%%%%%%%%%%%%%%%%%%%%%%%%%%%%%%%%%%%%%
%% For a section that needs to be included in the front matter insert into PreChapter.tex file. 
\newpage
\addcontentsline{toc}{chapter}{Front Matter Chapter}

% Abbreviations and such need to be in the front matter. The numbering needs to be roman numerals
%\begin{center}
\textbf{LIST OF ABBREVIATIONS}
\end{center}
\begin{multicols}{2}
\noindent 2D - Two-Dimensional\\
\noindent 3D - Three-Dimensional\\
\noindent DEM - Digital Elevation Model\\
\noindent DSLR - Digital Single Lens Reflex\\
\noindent GCP - Ground Control Point\\
\noindent GeoTIFF - Georeferenced Tagged Image File Format\\
\noindent GIS - Geographical Information System\\
\noindent GPR - Ground-Penetrating Radar\\
\noindent GPS - Global Positioning System\\
\noindent GSSI - Geophysical Survey Systems Incorporated\\
\noindent IMU - Inertial Measurement Unit\\
\noindent LIA - Little Ice Age\\
\noindent LiDAR - Light Detection And Ranging\\
\noindent LIS - Laurentide Ice Sheet\\
\noindent m.a.s.l. - Meters Above Mean Sea Level\\
\noindent MTL - Marine Transgression Line\\
\noindent radar - Radio Detection And Ranging\\
\noindent RGB - Red, Green, Blue\\
\noindent RMS - Root Mean Square\\
\noindent RTK - Real Time Kinematic\\
\noindent SfM - Structure from Motion\\
\noindent SLR - Sea Level Rise\\
\noindent SSS - Sidescan Sonar\\
\noindent w.e. - Water Equivalent\\
\end{multicols}
\endinput

% If you have other lists which need to be included they go here, possibly using the listof environment
%\begin{listof}{...}		% Replace the ... with name of the things being listed here
% Contents of list
%\end{listof}

% Sets the document spacing and pagestyle.  It is recommended that the `bottom' option be used. 
\mainmatter{bottom} 	

\endinput